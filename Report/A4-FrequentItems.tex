\documentclass[11pt]{article}

\usepackage{classDM17}
\usepackage{mathtools}
\DeclarePairedDelimiter\ceil{\lceil}{\rceil}
\DeclarePairedDelimiter\floor{\lfloor}{\rfloor}

\title{Asmt 4: Frequent Items}
\author{Gopal Menon\\Turn in through Canvas by 2:45pm: \\
Wednesday, March 22}
\date{}

\begin{document}
\maketitle



\section{Streaming Algorithms}

\paragraph{A: (20 points)} 
Run the Misra-Gries Algorithm (see \textbf{L11.3.1}) with $(k - 1) = 9$ counters on streams $S1$ and $S2$. Report the output of the counters at the end of the stream.


    \begin{table}[!h] 
    \centering
    \caption{Misra-Gries Counter Outputs for stream $S1$}
    \label{S1Misra}
    \begin{tabular}{K{1.5cm}K{1.5cm}K{1.5cm}K{1.5cm}K{1.5cm}K{1.5cm}K{1.5cm}}
      \hline
   c  & a &  b & o & v & f & p  \\
      \hline      
      $105715$ &   $195715$              &  $155715$  &          $2$             & $1$      & $1$    & $1$           \\
      \hline      
    \end{tabular}
    \end{table}

    \begin{table}[!h] 
    \centering
    \caption{Misra-Gries Counter Outputs for stream $S2$}
    \label{S2Misra}
    \begin{tabular}{K{1.5cm}K{1.5cm}K{1.5cm}K{1.5cm}K{1.5cm}K{1.5cm}K{1.5cm}K{1.5cm}}
      \hline
   b  & c &  a & h & l & j & w & r  \\
      \hline      
      $135715$ &   $175715$              &  $245715$  &          $1$             & $1$      & $1$    & $1$    & $1$       \\
      \hline      
    \end{tabular}
    \end{table}

In each stream, from just the counters, report how many objects might occur more than 20\% of the time, and which must occur more than 20\% of the time. 

For any item $q$, the actual frequency $f_q$ and the frequency $\hat{f_q}$ reported by the algorithm are related by the equation

$$
f_q - \frac{m}{k} \leq \hat{f_q}
$$

where $m=1,000,000$ is the size of the stream, and $k=10$ is the number of counters.\\
Substituting these values into the above equation, we get

\begin{equation*}
\begin{aligned}
f_q - \frac{1,000,000}{10} &\leq \hat{f_q}\\
f_q - 100,000 &\leq \hat{f_q}\\
f_q - \hat{f_q}  &\leq 100,000
\end{aligned}
\end{equation*}

This means that the maximum possible undercounting is by $\num[group-separator={,}]{100000}$. Given that $20\%$ of $\num[group-separator={,}]{1000000}$ is $\num[group-separator={,}]{200000}$, any label with count more than $\num[group-separator={,}]{200000}$, must occur more than $20\%$ of the time since over-counting is not possible. So label $a$ in stream $S2$ must occur more than $20\%$ of the time. Any label with count between $\num[group-separator={,}]{100000}$ and $\num[group-separator={,}]{200000}$ might occur more that $20\%$ of the time. So labels $a$, $b$ and $c$ in stream $S1$ and labels $b$ and $c$ in stream $S2$ might occur more than $20\%$ of the time.

\paragraph{B: (20 points)}  

\end{document}
