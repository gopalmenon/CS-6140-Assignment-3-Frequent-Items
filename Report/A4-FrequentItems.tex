\documentclass[11pt]{article}

\usepackage{classDM17}
\usepackage{mathtools}
\DeclarePairedDelimiter\ceil{\lceil}{\rceil}
\DeclarePairedDelimiter\floor{\lfloor}{\rfloor}

\title{Asmt 4: Frequent Items}
\author{Gopal Menon\\Turn in through Canvas by 2:45pm: \\
Wednesday, March 22}
\date{}

\begin{document}
\maketitle



\section{Streaming Algorithms}

\paragraph{A: (20 points)} 
Run the Misra-Gries Algorithm (see \textbf{L11.3.1}) with $(k - 1) = 9$ counters on streams $S1$ and $S2$. Report the output of the counters at the end of the stream.\\

The estimated counts for streams $S1$ and $S2$ are given in tables \ref{S1Misra} and \ref{S2Misra} respectively.

    \begin{table}[!h] 
    \centering
    \caption{Misra-Gries Counter Outputs for stream $S1$}
    \label{S1Misra}
    \begin{tabular}{K{1.5cm}K{1.5cm}K{1.5cm}K{1.5cm}K{1.5cm}K{1.5cm}K{1.5cm}}
      \hline
   c  & a &  b & o & v & f & p  \\
      \hline      
      $\num[group-separator={,}]{105715}$ &   $\num[group-separator={,}]{195715}$              &  $\num[group-separator={,}]{155715}$  &          $2$             & $1$      & $1$    & $1$           \\
      \hline      
    \end{tabular}
    \end{table}

    \begin{table}[!h] 
    \centering
    \caption{Misra-Gries Counter Outputs for stream $S2$}
    \label{S2Misra}
    \begin{tabular}{K{1.5cm}K{1.5cm}K{1.5cm}K{1.5cm}K{1.5cm}K{1.5cm}K{1.5cm}K{1.5cm}}
      \hline
   b  & c &  a & h & l & j & w & r  \\
      \hline      
      $\num[group-separator={,}]{135715}$ &   $\num[group-separator={,}]{175715}$              &  $\num[group-separator={,}]{245715}$  &          $1$             & $1$      & $1$    & $1$    & $1$       \\
      \hline      
    \end{tabular}
    \end{table}

In each stream, from just the counters, report how many objects might occur more than 20\% of the time, and which must occur more than 20\% of the time. \\

For any item $q$, the actual frequency $f_q$ and the frequency $\hat{f_q}$ reported by the algorithm are related by the inequality

$$
f_q - \frac{m}{k} \leq \hat{f_q}
$$

where $m=\num[group-separator={,}]{1000000}$ is the size of the stream, and $k=10$ is the number of counters.\\
Substituting these values into the above equation, we get

\begin{equation*}
\begin{aligned}
f_q - \frac{\num[group-separator={,}]{1000000}}{10} &\leq \hat{f_q}\\
f_q - \num[group-separator={,}]{100000} &\leq \hat{f_q}\\
f_q - \hat{f_q}  &\leq \num[group-separator={,}]{100000}
\end{aligned}
\end{equation*}

This means that the maximum possible undercounting is by $\num[group-separator={,}]{100000}$. Given that $20\%$ of $\num[group-separator={,}]{1000000}$ is $\num[group-separator={,}]{200000}$, any label with count more than $\num[group-separator={,}]{200000}$, must occur more than $20\%$ of the time since over-counting is not possible. So label $a$ in stream $S2$ must occur more than $20\%$ of the time. Any label with count between $\num[group-separator={,}]{100000}$ and $\num[group-separator={,}]{200000}$ might occur more that $20\%$ of the time. So labels $a$, $b$ and $c$ in stream $S1$ and labels $b$ and $c$ in stream $S2$ might occur more than $20\%$ of the time.

\paragraph{B: (20 points)}  
Build a Count-Min Sketch (see \textbf{L12.1.1}) with $k = 10$ counters using $t = 5$ hash functions. Run it on streams $S1$ and $S2$.\\
For both streams, report the estimated counts for objects $a$, $b$, and $c$. Just from the output of the sketch, which of these objects, with probably $1 - \delta = \frac{31}{32}$, might occur more than $20\%$ of the time?\\

The estimated counts for streams $S1$ and $S2$ are given in table \ref{S3CountMin}.

    \begin{table}[!h] 
    \centering
    \caption{Count-Min Sketch Counter Outputs}
    \label{S3CountMin}
    \begin{tabular}{K{1.5cm}K{1.5cm}K{1.5cm}K{1.5cm}}
      \hline
   Stream  & a &  b & c  \\
      \hline      
      $S1$ &   $\num[group-separator={,}]{250000}$              &  $\num[group-separator={,}]{243121}$  &          $\num[group-separator={,}]{160000}$      \\
      \hline      
      $S2$ &   $\num[group-separator={,}]{290000}$              &  $\num[group-separator={,}]{206917}$  &          $\num[group-separator={,}]{220000}$      \\
      \hline      
    \end{tabular}
    \end{table}

For any item $q$, the actual frequency $f_q$ and the frequency $\hat{f_q}$ reported by the algorithm are related by the PAC bound, 

$$
f_q \leq \hat{f_q} \leq f_q + \varepsilon m
$$

where the second inequality holds with a probability of $1-\delta$. The number of hash functions is related to $\delta$ by $t=\log_2\left(\frac{1}{\delta} \right)$ the number of counters per hash function is related to $\varepsilon$ by $k = \frac{2}{\varepsilon}$, and $m$ is the number of items in the stream. \\

Given that 

\begin{equation*}
\begin{aligned}
1 - \delta &= \frac{31}{32}\\
\delta &= 1 - \frac{31}{32}\\
&= \frac{1}{32}\\
t&=\log_2\left(\frac{1}{\delta} \right)\\
&= \log_2\left(\frac{1}{\frac{1}{32}} \right)\\
&= \log_2\left(32\right)\\
&=5
\end{aligned}
\end{equation*}

This value of $t$ matches with the number of hash functions in the experiment.

\begin{equation*}
\begin{aligned}
k &= \frac{2}{\varepsilon}\\
\varepsilon &= \frac{2}{k}\\
&=\frac{2}{10}
\end{aligned}
\end{equation*}

From the above inequality, we can see that 

\begin{equation*}
\begin{aligned}
\hat{f_q} &\leq f_q + \varepsilon m\\
&\leq f_q + \frac{2}{10} \times \num[group-separator={,}]{1000000}\\
&\leq f_q + \num[group-separator={,}]{200000}\\
\hat{f_q} - f_q  &\leq \num[group-separator={,}]{200000}
\end{aligned}
\end{equation*}

This means that the over-counting is limited by $\num[group-separator={,}]{200000}$. Since under-counting is not possible, we can see that since $a$ and $b$ in stream $S1$ and $a$, $b$ and $c$ in stream $S2$ have an estimated count of at least $\num[group-separator={,}]{200000}$, these labels might occur more that $20\%$ of the time (since $20\%$ of $\num[group-separator={,}]{1000000}$ is $\num[group-separator={,}]{200000}$) with probability $\frac{31}{32}$.

\paragraph{C: (5 points)} 
How would your implementation of these algorithms need to change (to answer the same questions) if each object of the stream was a \enquote{word} seen on Twitter, and the stream contained all tweets concatenated together?

\paragraph{D: (5 points)} 
Describe one advantage of the Count-Min Sketch over the Misra-Gries Algorithm.

\end{document}
